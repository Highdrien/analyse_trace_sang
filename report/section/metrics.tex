\subsection{Evaluation des modèles}
\label{sec: metrics}
\subsubsection{Accuracy micro}
L'accuracy est le pourcentage de réussite d'un modèle. Elle est calculée par la formule suivante :
\begin{equation}
    \text{Accuracy micro} = \frac{\text{Nombre de prédictions correctes}}{\text{Nombre total de prédictions}}
\end{equation}

\subsubsection{Accuracy macro}
L'accuracy macro est la moyenne des accuracy de chaque classe. Elle est calculée par la formule suivante :
\begin{equation}
    \text{Accuracy macro} = \frac{1}{N} \sum_{i=1}^{n} \frac{\text{Nombre de prédiction correctes de la classe } i}{\text{Nombre total de prédictions de la classe } i}{}
\end{equation}

\subsubsection{Précision}
La précision est le pourcentage de prédictions correctes parmi les prédictions positives. Elle est calculée par la formule suivante :
\begin{equation}
    \text{Précision} = \frac{\text{Nombre de vrais positifs}}{\text{Nombre de vrais positifs + Nombre de faux positifs}}
\end{equation}

\subsubsection{Rappel}
Le rappel est le pourcentage de prédictions correctes parmi les vrais labels positifs. Il est calculé par la formule suivante :
\begin{equation}
    \text{Rappel} = \frac{\text{Nombre de vrais positifs}}{\text{Nombre de vrais positifs + Nombre de faux négatifs}} 
\end{equation}

\subsubsection{F1-score}
Le F1-score est la moyenne harmonique de la précision et du rappel. Il est calculé par la formule suivante :
\begin{equation}
    \text{F1-score} = 2 \times \frac{\text{Précision} \times \text{Rappel}}{\text{Précision} + \text{Rappel}}
\end{equation}

\subsubsection{Top 3}
Le Top 3 est le pourcentage de trouver la réponse parmi les 3 classes les plus prédites par le modèle. Il est calculé par la formule suivante :
\begin{equation}
    \text{Top 3} = \frac{\text{Nombre de prédictions correctes parmi les 3 premières prédictions}}{\text{Nombre total d'exemples}}
\end{equation}

\subsection{Evaluation des cartes de saillance}
\label{sec: grad metrics}

Soit $p_i = \text{arg}\max_c p_i^c$ la probabilité de la classe prédicte par le modèle pour l'image $x_i$. Soit $o_i$ la probabilité de cette même classe prédite par le modèle pour l'image maské de ca carte de saillance $x_i * s_i$ (où $s_i$ est la carte de saillance de l'image $x_i$). On définie $N$ le nombre totale d'image dans le test set. On note $[x]_+$ le maximun entre $0$ et la valeur $x$ ($[x]_+=\max(0, x)$). On définie les métriques suivantes :

\subsubsection{Averare Drop (AD)}
L'Average Drop (AD) quantifie la perte de pouvoir prédictif, mesurée en termes de probabilité de classe, lorsque nous masquons uniquement l'image image. Elle est comprise entre 0 et 100, et une valeur plus faible est meilleure.

\begin{equation}
    \text{AD} = \frac{1}{N} \sum_{i=1}^{N} \frac{[p_i - o_i]_+}{p_i} \cdot 100
\end{equation}

\subsubsection{Average Increase (AI)}
L'Average increase (AI) également connu sous le nom d'increase in confidence, mesure le pourcentage d'images pour lesquelles l'image masquée donne une probabilité de classe plus élevée que l'image originale. Elle est compris en 0 et 100, et une valeur plus haute est la meilleur.

\begin{equation}
    \text{AI} = \frac{1}{N} \sum_{i=1}^{N} 1_{p_i<o_i} \cdot 100
\end{equation}

\subsubsection{Average Gain (AG)}
L'Average Gain (AG) quantifie le pouvoir prédictif, mesuré en tant que probabilité de classe, lorsque nous masquons l'image. Elle est comprise entre 0 et 100, et une valeur plus élevée est meilleure.
\begin{equation}
    \text{AG} = \frac{1}{N} \sum_{i=1}^{N} \frac{[o_i - p_i]_+}{1 - p_i} \cdot 100
\end{equation}

