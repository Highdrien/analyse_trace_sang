
Les résultats de test sur les données de laboratoire présent dans la Table~\ref{tab:results_lab} indiquent de meilleures performances pour le modèle Resnet qui a été réentraîné sur tous ses poids, selon l'accuracy et le F1-score. 

\begin{table}[ht]
  \centering
    \begin{tabular}{cccccc}
    \toprule
    Modèles & Acc Micro & Acc Macro & F1-score & Top 3 \\
    \midrule
    LP ResNet & 95.2 & 94.3 & 94.7 & \textbf{99.8} \\
    AWL ResNet & \textbf{97.3} & \textbf{97.1} & \textbf{96.2} & \textbf{99.8} \\
    Adversarial & 93.4 & 91.8 & 91.8 & \textbf{99.8} \\
    \bottomrule
    \end{tabular}
    \caption{Résultats de test sur les données de laboratoire}
    \label{tab:results_lab}
\end{table}

Les résultats de test sur les données de laboratoire présent dans la Table~\ref{tab: results_real} indiquent de meilleures performances pour le modèle Resnet qui était entraîné entièrement sur les données de laboratoire, puis fine-tuné sur les données réelles, selon l'accuracy et le F1-score. 
Nous avons également une valeur de Top 3 à 51.7\%, ce qui signifie qu'il y a une chance sur deux que la bonne réponse soit présente dans la prédiction. 

\begin{table}[ht]
  \centering
    \begin{tabular}{ccccccc}
    \toprule
    Modèles & Acc Micro & Acc Macro & F1-score & Top 3 \\
    \midrule
    LP ResNet & 12.9 & 6.0 & 4.0 & 19.0 \\
    FT LP ResNet & 11.8 & 6.1 & 6.4 & 26.0 \\
    AWL Resnet & 17.2 & 13.8 & 8.1 & 28.7 \\
    FT AWL ResNet & \textbf{41.9} & \textbf{33.4} & \textbf{26.9} & \textbf{51.7} \\
    Adversarial & 11.8 & 5.7 & 3.7 & 16.7 \\
    \bottomrule
    \end{tabular}
    \caption{Résultats de test sur les données réelles}
    \label{tab: results_real}
\end{table}

Le modèle Adversarial a des performances assez décevantes par rapport à un modèle Resnet, donc le modèle Adversarial n'est pas le plus adapté pour notre problématique.

Le AWL Resnet a étonnament de meilleures performances que le modèle Resnet appliquant le Linear Probing. 
